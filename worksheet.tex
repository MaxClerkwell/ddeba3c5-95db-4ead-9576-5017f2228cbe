\documentclass{dcbl/challenge}

\setdoctitle{Shell-Scripts}
\setdocauthor{Stephan Bökelmann}
\setdocemail{sboekelmann@ep1.rub.de}
\setdocinstitute{AG Physik der Hadronen und Kerne}

\usepackage{hyperref}
\usepackage{listings}


\begin{document}

The term operating system meant everything and nothing in the days before Unix.
Due to this, the term OS wasn't actually used that much in reference to Unix in the beginning, but rather the term programming-environment.
Manipulating streams of data is one of the most commonly done tasks in programming.
A whole plathora of stream manipulations can be done in just a single line of shell-commands, but in order to keep them reproducible, we'd like to save them for later.
This is what shell-scripts are for and thus we will also start our programming journey with shell-scripts.


\section*{Exercises}
\begin{aufgabe}
    Navigate to your home directory and run create a new directory called \texttt{myScripts}.
    Navigate into this directory and create a new file called \texttt{hello}.
    Open this file and add the following content: \texttt{echo "Hello World!"}.
    Save and close the file.
    Run the file by executing the following command \texttt{bash hello}.
    We run a new shell explicitly by calling \texttt{bash} and hand it the name of the file \texttt{hello} as a parameter.
\end{aufgabe}

\begin{aufgabe}
    Since we already run a shell-session, we don't always want to write the \texttt{bash} command every time.
    Instead, we can make sure, that our shell understands, that this is an executable program, by adding a line to the file and changing it's metainformation.
    Open the script file and add the following content to be the first line: \texttt{\#!/bin/bash}.
    The so called \textit{shebang} tells the shell, which program is able to execute this script.
    Save and close the file.
    Furthermore, we have to make this script executable by changing the metainformation.
    Run \texttt{ls -ls > before} in the directory, then \texttt{chmod +x hello}, then \texttt{ls -la > after}. 
    Use \texttt{vimdiff before after} to inspect the differences.
    Close \texttt{vimdiff} by using the \texttt{:q}-command two times.
    Try running \texttt{./hello}.
    Now we don't need to type the \texttt{bash} command every time.
\end{aufgabe}

\begin{aufgabe}
    This works quite well for us, but we'd like to get rid of the \texttt{./} in front of our command. 
    Use the command \texttt{echo \$PATH}, to print the so called path-variable to your console.
    \texttt{\$PATH} is the variable that contains all the directories, where executable scripts might be found automatically by the shell.
    Let's add our current path to this environment-variable to be able to run our script without the need of \texttt{./}.
    Therefore we first save our current path in an extra variable called \texttt{scriptpath} by running \texttt{export SCRIPTPATH=\$(pwd)}. 
    This will execute the command \texttt{pwd} and save the result in the variable \texttt{scriptpath}.
    Look at the variable by running \texttt{echo \${scriptpath}}.
    Now we want to make sure, that our script-path is appended to the actual path.
    Therefore we run \texttt{export PATH=\$PATH:\$SCRIPTPATH}. 
    The colon is used as a separator between the paths in the variable.
    We set the new path to be the old path plus the scriptpath, and print it out again.
    Let's see the result by running \texttt{echo \$PATH}.
    Try running \texttt{hello} without the need of \texttt{./}.
\end{aufgabe}

\begin{aufgabe}
    Close your shell and open a new one.
    Print your path-variable by running \texttt{echo \$PATH}.
    You will see, that the changes have been lost. 
    They were only added in a temporary session. 
    In order to set the path permanently, we need to modify the standard configuration of our shell-sessions. 
    This is done by editing the file \texttt{$\sim$/.bashrc}.
    Open the file in a text-editor and add the following line at the bottom: \texttt{export PATH=\$PATH:<actual scriptpath>}.
    Make sure to replace <actual scriptpath> with the path where the script is located.
    \texttt{$\sim$/.bashrc} is the so called run-command-file, which is stored in the user's home-directory and will be loaded on every shell-session of this particular user.
\end{aufgabe}

\begin{aufgabe}
    We may also call a shell-script with arguments. 
    Go back to your directory \texttt{myScripts} and create a new file called \texttt{sum}.
    Open the file and add the following content:\\
    \begin{lstlisting}[language=bash, caption=Beispiel Shell-Script]
        #!/bin/bash
        sum=$(($1+$2))
        echo $sum
    \end{lstlisting}
    Save and close the file.
    Run \texttt{./sum 3 5} to see the result.
\end{aufgabe}

\section*{Annotations}
\begin{enumerate}
    \item Book - The Unix Programming Environment: \href{https://www.amazon.com/-/de/dp/013937681X/ref=sr_1_1?crid=ZKV5OYC7B2XL&dib=eyJ2IjoiMSJ9.v5CMlrbXxKNhJIQ9iZvGeMZy7oFDdqBAnw6zl2-EzGCqPGxoH4dg4ufJ-uftq4__sS0dCnpEB3K6Ww_pbnsU57rcy8_4TXWXlW46DnHlNz3fRD7c1IKeLKYl93tlUhbze1Ll_8DWYLieGNRjMVBKfgPas9NEc79Q4YjI42-E7COCtA0TdsFaelwsNb-URkEx7sGlIM5XiiyM5YNLV4c90m2HcTzFKltwxrYPxnpVXhw.mRUMSZNTrIguh4aCuI0kzn2jBvh6KKybz4atkIP2hxw&dib_tag=se&keywords=unix+programming+environment&qid=1711993814&sprefix=unix+programmi\%2Caps\%2C313&sr=8-1}{Amazon-Link}
\end{enumerate}

\end{document}
